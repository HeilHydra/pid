\section{Literature Review Outline}

Below, we have identified literature that will be looked at and discussed in order to feed the requirements stage with valuable information.
\begin{itemize}
  \item \textbf{The Internet of Things} - 
    The Internet of Things (IoT) is something that is spoken more frequently. It is the connection of identifiable devices within an existing internet infrastructure. It allows for data transfer across a network without human-to-human interaction. A detailed look into this will determine how we connect people with music in a very efficient and sleek manner.
  \item \textbf{Competitor Analysis} - 
    We have taken a look at other apps and services on the market and feel that nothing matches our idea. The closest service available is from \textbf{Sonos}. Sonos is a smart system of speakers and audio components that unite your digital music collection in one app and can then be controlled from any device. The rise of digital music has allowed for us to bring our music wherever we go, through media such as iPods, MP3 players etc. However, there hasnt been the same advancements in terms of systems that don't move around. `Wireless' is the word that comes to mind when thinking of a solution. It is now possible to stream audio to a wireless device (speaker) and without compromising on the sound quality. Sonos provides the user with the ability to play music from a device wirelessly anywhere in the \textbf{home}. However, there are two issues with this; it is only available for the \textbf{home} and you need their \textbf{hardware} which is not cheap. The cheapest speaker you can purchase is \pounds169. If you are a music-orientated person, you may wish to purchase their high-end hardware which can cost anything up to \pounds1200. Unfortunately, Sonos does not support other wireless speakers. An adaptor can be purchased to allow these to be connected to their system, the \emph{Connect} device, but that costs \pounds279.  

    \textbf{Pure} have also moved into this market where they provide wireless speakers and hardware for \emph{wireless music} in the \textbf{home}. They too allow you to purchase hardware to link your current speakers with their system and at \pounds69, it is much cheaper than Sonos but is still not cheap. It allows you to wirelessly play your music from any music app or streaming service you want. \textbf{Bose} also provide a very similar service to Pure, but the hardware costs are more expensive.

    From this, we can see that there are no services available for the wireless sharing of music outside the home. Our app would unite music into your whole day routine, whether it be at the office, coffee shop, restaurant as well as the home. We also want to make sure that no expensive costs are applied.
    Competitors can appear at anytime during the development of a project, so it is important that we keep looking for emerging customers and that we can identfy how our product is unique to theirs. 

  \item \textbf{Music Sources} -
    In today's maket, there are many music sources available to an individual. We have virtual music from services such as `Spotify', `Google Music' and `iTunes' as well as physical music on `iPods', `MP3 players' and other hardware devices. 

    \textbf{Spotify} is a music streaming service that offers access to a library of over 20 million music tracks with over 40 million active users. It is available across 58 markets including the UK, USA, France, Germany, Hong Kong and Argentina. It is available on iOS, Andriod, Windows as well as PC and Mac. One chain that is affiliated with Spotfiy is Costa. Costa have their own playlist that people can access frm their device. 
    \textbf{Google Music} is another streaming service and offers the same service as Spotify. Again, they have a large library of songs (around 18 million) and the service is available in over 57 countries on all android devices as well as web browsers. 
    \textbf{iTunes} A year ago, Apple's iTunes accounted for 75\% of the digital music market and with a huge 575 million active users. Although this may have decreased slightly in the last year, that is still a huge user base. As well as general users, the likes of Starbucks are affiliated with iTunes and use this service to hand-pick and play music throughout their stores. The idea of allowing customers to put forward their music preference may be of interest to a chain like Starbucks amongst others. 

    The above figures suggest that the music streaming industry is vast and that music is a part of many people's lives. Coffee shops have integrated these sources and music into their environment, but without the customer interaction. Choona would provide this interaction. Over the course of this project, we shall identify more sources because having more sources creates a larger user base as well as a better music library. We shall look at how the different sources work to try and make sure we have adaptors in place that can cover the wide variety of sources available. 

  \item \textbf{Legal Issues} - 
    Music playing for customers or staff through media such as radio, MP3, TV etc. is considered a \emph{public performance}. The \emph{Copyright, Designs and Patents Act 1988} means that an agreement is needed from the copyright owner before the material can be played in public. A music license (PPL) will grant this agreement. In most cases, a license is required but there are a few instances when one is not required. One example of this is where PRS (PRS for Music represents the rights of over 100,000 artists in the UK, licensing organisations to play, perform or make available copyright music on behalf of our artists and overseas societies, distributing the royalties to them fairly and efficiently) artists have waived their rights. Another example is a hotel, guest house or B\&B that has fewer than 25 rooms with no areas open to non-residents. 
    Any business such as a coffee shop, bar or gym that plays recorded music in public will legally require a PPL. The likelihood of our service being used in places that don't have a PPL and require one is small. Most coffee shops, restaurants, gyms etc. will already have the license in place. It will be work places deciding to implement our service that will have to go about retrieving a PPL. As part of this section, we shall look at the process of obtaining a PPL, the costs involved and potential constraints.

  \item \textbf{NFC} - 
    Near-Field Communication (NFC) is a form of short range wireless communication (4cm or less to launch a connection) allowing for radio communication and the passing of data packets between two devices, or smart tags that work with NFC. NFC is an advancement to RFID systems because it allows for two-way communication. This two way communication can then be used for authentication purposes or two-way communication. 
    At the moment, not all devices support NFC and it is suggested around 20\% of phones worldwide will have NFC capability by the end of 2014. This figure is set to increase dramatically over the next five years meaning more devices will have the capability and more users will be familiar with the technology. This widespread reach of NFC phones could mean one day that NFC tags become as commonplace as bar codes, so it makes sense to make use of this technology.

    Using NFC tags with a mobile device, a user could access the playlist at their current location without the need to search for it. The idea is very durable as NFC tags are small and cheap enough to integrate anywhere. They do not need a power source but instead draw power from the device that reads them. 
    We shall look in more detail at how we can install these into facilities, the upgrading process and determine whether they can be managed internally by the business themselves or will outside assistance be required.
\end{itemize}