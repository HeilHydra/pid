\section{Risk Assessment}

With any project, regardless of type or size there are always risks involved. A risk assessment needs to be done to see what's involved, whose effected and what needs to be done in case the risk takes place. This is done beforehand in project planning to reduce the overall risk of the project. Furthermore if any risk is enabled, it can be dealt with quickly and effectively. The following table has been formulated to identify the risks associated with this project and how each risk can be reduced.

\begin{center}

\begin{table}[h]
\begin{tabular}{|l|l|l|l|l|l|}
\hline
\multicolumn{1}{|c|}{\textbf{Risk}}         & \multicolumn{1}{c|}{\textbf{Probability}} & \multicolumn{1}{c|}{\textbf{Impact}} & \multicolumn{1}{c|}{\textbf{Effect}}                                                                                                                                & \multicolumn{1}{c|}{\textbf{Risk reduction actions}}                                                                                                 & \multicolumn{1}{c|}{\textbf{If it happens; triggers and actions}}                                                                                                                                                                                              \\ \hline
Poor capture of user requirements           & High                                      & High                                 & \begin{tabular}[c]{@{}l@{}}Possible failure/delay of delivering solution to Atos on time.\\ Re-capture requirements; waste of time and resources.\end{tabular}      & Ensure user requirements are understood and captured properly at the very start of the project.                                                      & \begin{tabular}[c]{@{}l@{}}Triggers: If the initial feedback requests a lot of changes.\\ Actions: Re-capture requirements using feedback.\end{tabular}                                                                                                        \\ \hline
Significant changes to the requirements     & Low                                       & High                                 & \begin{tabular}[c]{@{}l@{}}Rethinking concepts around the idea.\\ Waste of time and resources.\end{tabular}                                                         & Ensure requirements are collected properly and both parties are on the same page.                                                                    & \begin{tabular}[c]{@{}l@{}}Triggers: Significant changes to the specification.\\ Actions: Discuss changes between the team and what effect it has to other phases if implemented.\end{tabular}                                                                 \\ \hline
Poor communication between the team members & Low                                       & Medium                               & \begin{tabular}[c]{@{}l@{}}Duplication of work.\\ Team is unsure on what needs to be done next.\\ Waste of time and resources.\end{tabular}                         & Pre-plan what needs to be done within the project and ensure everyone is aware of what others are doing.                                             & \begin{tabular}[c]{@{}l@{}}Triggers: Duplication of work and team members unsure on what they are doing.\\ Actions: Gather group, discuss roles and assign tasks. This needs to be an ongoing process.\end{tabular}                                            \\ \hline
Loss of data                                & Low                                       & High                                 & \begin{tabular}[c]{@{}l@{}}Systems may lose data parity, causing software errors.\\ Users may receive a bad experience if work they have done is lost.\end{tabular} & Database redundancy can be used to provide backups. Periodic database dumps can be used for recovery in case of catastrophic infrastructure failure. & \begin{tabular}[c]{@{}l@{}}Triggers: Infrastructure failure.\\ Actions: Ensure backup databases take over, if not try and recovered from periodic backup.\end{tabular}                                                                                         \\ \hline
Infrastructure failure                      & Low                                       & High                                 & \begin{tabular}[c]{@{}l@{}}System outages.\\ Data loss.\end{tabular}                                                                                                & Server redundancy, automatic failover and load balancing can all be used to minimize the risk and help recover from infrastructure failures.         & \begin{tabular}[c]{@{}l@{}}Triggers: Power failure, DoS attacks, network issues, software failures.\\ Actions: Ensure backup infrastructure automatically takes over, if backup fails then manually bring up services to ensure system stability.\end{tabular} \\ \hline
\end{tabular}
\end{table}

\end{center}